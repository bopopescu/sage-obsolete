\documentclass[12pt]{article}

\usepackage{amssymb, amsmath}

\textwidth=6in
\textheight=8.9in
\topmargin=-0.5in
\evensidemargin=0.25in
\oddsidemargin=0.25in

\newtheorem{theorem}{Theorem}[section]
\newtheorem{definition}[theorem]{Definition}

\title{Tutorial for p-adics in SAGE}
\author{David Roe}
\date{\today}

\begin{document}

\def\ZZ{\mathbb{Z}}
\def\QQ{\mathbb{Q}}
\def\Qp{\mathbb{Q}_p}
\def\Zp{\mathbb{Z}_p}
\def\Zpx{\mathbb{Z}_p^{\times}}
\def\Zpn{\mathbb{Z} / p^n\mathbb{Z}}
\def\OK{\mathcal{O}_K}

\maketitle

\section{Introduction}

$p$-adics in SAGE are currently undergoing a transformation.
Previously, SAGE has included a single class representing $\Qp$,
and a single class representing elements of $\Qp$.
Our goal is to create a rich structure of different options
that will reflect the mathematical structures of the $p$-adics.
This is very much a work in progress: some of the classes
that we eventually intend to include have not yet been written,
and some of the functionality for classes in existence has not yet been implemented.
In addition, while we strive for perfect code,
bugs (both subtle and not-so-subtle) continue to evade our clutches.
As a user, you serve an important role.
By writing non-trivial code that uses the $p$-adics,
you both give us insight into what features are actually used
and also expose problems in the code for us to fix.

Our design philosophy has been to get a robust,
usable interface working first,
with simpleminded implementations underneath.
We want this interface to stabilize rapidly,
so that users' code does not have to change.
Once we get the framework in place,
we can go back and work on the algorithms and implementations underneath.
All of the current $p$-adic code is currently written in pure Python,
which means that it does not have the speed advantage of compiled code.
Thus our $p$-adics can be painfully slow at times when you're doing real computations.
However, finding and fixing bugs in Python code is \emph{far} easier
than finding and fixing errors in the compiled alternative within SAGE (SageX),
and Python code is also faster and easier to write.
We thus have significantly more functionality implemented and working
than we would have if we had chosen to focus initially on speed.
And at some point in the future, we will go back and improve the speed.
Any code you have written on top of our $p$-adics will then
get an immediate performance enhancement.

If you do find bugs, have feature requests or general comments, please let me know
at roed@math.harvard.edu.

This tutorial attempts to outline what you need to know in order to use
the $p$-adics effectively.  OUTLINE SECTIONS.

\section{Terminology and types of $p$-adics}

To write down a $p$-adic element completely would require an infinite amount of data.
Since computers do not have infinite storage space, we must instead store finite
approximations to elements.  Thus, just as in the case of floating point numbers for
representing reals, we have to store an element to a finite precision level.
The different ways of doing this account for the different types of $p$-adics.

We can think of $p$-adics in two ways.  First, as a projective limit of finite groups:
$$\Zp = \lim_{\leftarrow n} \Zpn.$$
Secondly, as Cauchy sequences of rationals (or integers, in the case of $\Zp$, under the
$p$-adic metric.  Since we only need to consider these sequences up to equivalence, this
second way of thinking of the $p$-adics is the same as considering power series in $p$ with
integral coefficients in the range $0$ to $p-1$.  If we only allow nonnegative powers of $p$
then these power series converge to elements of $\Zp$, and if we allow bounded negative powers
of $p$ then we get $\Qp$.

Both of these representations give a natural way of thinking about finite approximations to a
$p$-adic element.  In the first representation, we can just stop at some point
in the projective limit, giving an element of $\Zpn$.  As $\Zp / p^n\Zp \cong \Zpn$,
this is is equivalent to specifying our element modulo $p^n\Zp$.
\begin{definition}
The \emph{absolute precision} of a finite approximation $\bar{x} \in \Zpn$ to $x \in \Zp$
is the non-negative integer $n$.
\end{definition}
In the second representation, we can achieve the same thing by truncating a series 
$$a_0 + a_1 p + a_2 p^2 + \cdots$$
at $p^n$, yielding
$$a_0 + a_1 p + \cdots + a_{n-1} p^{n-1} + O(p^n).$$
As above, we call this $n$ the absolute precision of our element.

Given any $x \in \Qp$ with $x \ne 0$, we can write $x = p^v u$ where $v \in \ZZ$ and $u \in Zpx$.
We could thus also store an element of $\Qp$ (or $\Zp$) by storing $v$ and a finite approximation
of $u$.  This motivates the following definition:
\begin{definition}
The \emph{relative precision} of an approximation to $x$ is defined as the absolute precision
of the approximation minus the valuation of $x$.
\end{definition}
For example, if $x = a_k p^k + a_{k+1} p^{k+1} + \cdots + a_{n-1} p^{n-1} + O(p^n)$  then the
absolute precision of $x$ is $n$, the valuation of $x$ is $k$ and the relative precision of $x$
is $n-k$.

There are four different representations of $\Zp$ in Sage and two representations of $\Qp$:
the fixed modulus ring, the capped absolute precision ring, the capped relative precision
ring, the capped relative precision field, the lazy ring and the lazy field.

\subsection{Fixed Modulus Rings}
The first, and simplest, type of $\Zp$ is basically a wrapper around $\Zpn$, providing a
unified interface with the rest of the $p$-adics.  You specify a precision, and all elements
are stored to that absolute precision.  If you perform an operation that would normally lose
precision, the element does not track that it no longer has full precision.

The fixed modulus ring provide the lowest level of convenience, but it is also the one that
has the lowest computational overhead.  Once we have ironed out some bugs, the fixed modulus
elements will be those most optimized for speed.

As with all of the implementations of $\Zp$, one creates a new ring using the constructor
\verb/Zp/, and passing in \verb/'fixed-mod'/ for the \verb/type/ parameter.  For example,
\begin{verbatim}
sage: R = Zp(5, prec = 10, type = 'fixed-mod', print_mode = 'series')
sage: R
5-adic Ring of fixed modulus 5^10
\end{verbatim}

One can create elements as follows:
\begin{verbatim}
sage: a = R(375)
sage: a
3*5^3 + O(5^10)
sage: b = R(105)
sage: b
5 + 4*5^2 + O(5^10)
\end{verbatim}

Now that we have some elements, we can do arithmetic in the ring.
\begin{verbatim}
sage: a + b
5 + 4*5^2 + 3*5^3 + O(5^10)
sage: a * b
3*5^4 + 2*5^5 + 2*5^6 + O(5^10)
sage: a // 5
3*5^2 + O(5^10)
\end{verbatim}

Since elements don't actually store their actual precision, one can only divide by units:
\begin{verbatim}
sage: a / 2
4*5^3 + 2*5^4 + 2*5^5 + 2*5^6 + 2*5^7 + 2*5^8 + 2*5^9 + O(5^10)
sage: a / b
...
<type 'exceptions.ValueError'>: cannot invert non-unit
\end{verbatim}

If you want to divide by a non-unit, do it using the \verb@//@ operator:
\begin{verbatim}
sage: a // b
3*5^2 + 3*5^3 + 2*5^5 + 5^6 + 4*5^7 + 2*5^8 + O(5^10)
\end{verbatim}

\subsection{Capped Absolute Rings}
The second type of implementation of $\Zp$ is similar to the fixed modulus implementation,
except that individual elements track their known precision. 
The absolute precision of each element is limited to be less than the precision cap of the ring,
even if mathematically the precision of the element would be known to greater precision
(see Appendix A for the reasons for the existence of a precision cap).

Once again, use \verb/Zp/ to create a capped absolute $p$-adic ring.
\begin{verbatim}
sage: R = Zp(5, prec = 10, type = 'capped-abs', print_mode = 'series')
sage: R
5-adic Ring with capped absolute precision 10
\end{verbatim}

We can do similar things as in the fixed modulus case:
\begin{verbatim}
sage: a = R(375)
sage: a
3*5^3 + O(5^10)
sage: b = R(105)
sage: b
5 + 4*5^2 + O(5^10)
sage: a + b
5 + 4*5^2 + 3*5^3 + O(5^10)
sage: a * b
3*5^4 + 2*5^5 + 2*5^6 + O(5^10)
sage: c = a // 5
sage: c
3*5^2 + O(5^9)
\end{verbatim}

Note that when we divided by 5, the precision of \verb/c/ dropped.  This lower precision is now reflected in arithmetic.
\begin{verbatim}
sage: c + b
5 + 2*5^2 + 5^3 + O(5^9)
\end{verbatim}

Division is allowed: the element that results is a capped relative field element, which is discussed in the next section:
\begin{verbatim}
sage: 1 / (c + b)
5^-1 + 3 + 2*5 + 5^2 + 4*5^3 + 4*5^4 + 3*5^6 + O(5^7)
\end{verbatim}

\subsection{Capped Relative Rings and Fields}
Instead of restricting the absolute precision of elements (which doesn't make much sense when elements have negative
valuations), one can cap the relative precision of elements.  This is analogous to floating point representations
of real numbers.  As in the reals, multiplication works very well: the valuations add and the relative precision of
the product is the minimum of the relative precisions of the inputs.   Addition, however, faces similar issues as
floating point addition: relative precision is lost when lower order terms cancel.

To create a capped relative precision ring, use \verb/Zp/ as before.  To create capped relative precision fields, use
\verb/Qp/.
\begin{verbatim}
sage: R = Zp(5, prec = 10, type = 'capped-rel', print_mode = 'series')
sage: R
5-adic Ring with capped relative precision 10
sage: K = Qp(5, prec = 10, type = 'capped-rel', print_mode = 'series')
sage: K
5-adic Field with capped relative precision 10
\end{verbatim}

We can do all of the same operations as in the other two cases, but precision works a bit differently:
the maximum precision of an element is limited by the precision cap of the ring.
\begin{verbatim}
sage: a = R(375)
sage: a
3*5^3 + O(5^13)
sage: b = K(105)
sage: b
5 + 4*5^2 + O(5^11)
sage: a + b
5 + 4*5^2 + 3*5^3 + O(5^11)
sage: a * b
3*5^4 + 2*5^5 + 2*5^6 + O(5^14)
sage: c = a // 5
sage: c
3*5^2 + O(5^12)
sage: c + 1
1 + 3*5^2 + O(5^10)
\end{verbatim}

As with the capped absolute precision rings, we can divide, yielding a capped relative precision field element.
\begin{verbatim}
sage: 1 / (c + b)
5^-1 + 3 + 2*5 + 5^2 + 4*5^3 + 4*5^4 + 3*5^6 + 2*5^7 + 5^8 + O(5^9)
\end{verbatim}

\subsection{Lazy Rings and Fields}
The model for lazy elements is quite different from any of the other types of $p$-adics.
In addition to storing a finite approximation, one also stores a method for increasing
the precision.  The interface supports two ways to do this: \verb/set_precision_relative/ and 
\verb/set_precision_absolute/.

\begin{verbatim}
sage: R = Zp(5, prec = 10, type = 'lazy', print_mode = 'series', halt = 30)
sage: R
Lazy 5-adic Ring
sage: R.precision_cap()
10
sage: R.halting_parameter()
30
sage: K = Qp(5, type = 'lazy')
sage: K.precision_cap()
20
sage: K.halting_parameter()
40
\end{verbatim}

There are two parameters that are set at the creation of a lazy ring or field.  The first is \verb/prec/, which
controls the precision to which elements are initially computed.  When computing with lazy rings, sometimes situations
arise where the insolvability of the halting problem gives us problems.  For example,
\begin{verbatim}
sage: a = R(16)
sage: b = a.log().exp() - a
sage: b
O(5^10)
sage: b.valuation()
...
<class 'sage.rings.padics.precision_error.HaltingError'>: Stopped computing sum: set halting paramter higher if you want computation to continue

The second is \verb/halt/

\begin{verbatim}


\end{document}